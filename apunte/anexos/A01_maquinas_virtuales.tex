%
% Apunte de Sistemas Operativos
% Copyright (C) 2014 Esteban De La Fuente Rubio (esteban[at]delaf.cl)
%
% Permission is granted to copy, distribute and/or modify this document
% under the terms of the GNU Free Documentation License, Version 1.3
% or any later version published by the Free Software Foundation;
% with no Invariant Sections, no Front-Cover Texts, and no Back-Cover Texts.
% A copy of the license is included in the section entitled "GNU
% Free Documentation License".
%
% Link: http://www.gnu.org/copyleft/fdl.html
%

% MÁQUINAS VIRTUALES
\chapter{Máquinas virtuales}
Una máquina virtual entrega una abstracción del hardware de la máquina hacia el
sistema operativo, proporcionando una interfaz de hardware virtual similar a la
de la máquina real. Los discos duros son emulados, por ejemplo, mediante
imágenes de discos. El sistema operativo que corre sobre la máquina virtual
desconoce tal condición, o sea, no sabe que funciona sobre una máquina virtual y
no una real. Este tipo de sistemas permite correr múltiples sistemas operativos
sobre una misma máquina. Ejemplos de sistemas de virtualización son KVM, XEN y
VirtualBox.

El sistema operativo que corre sobre la máquina virtual también posee un modo de
ejecución usuario y de sistema, sin embargo estos son modos virtuales que corren
sobre un modo usuario real. Esto significa que si en el sistema operativo
virtual hay una solicitud a una llamada del sistema a través de un programa que
corre en modo usuario virtual, esta será procesada por el sistema operativo en
modo sistema virtual y se entregará a la máquina virtual, la cual, en modo
usuario real, atenderá la interrupción mediante el hardware virtualizado y
entregará la respuesta al sistema operativo. En caso que se requiera acceso al
hardware real, la máquina virtual deberá hacer uso de la API del sistema
operativo real para acceder al recurso solicitado.

Es importante mencionar que los tiempos de respuesta en máquinas virtuales serán
más lentos que en máquinas reales. Lo anterior debido a la emulación que se debe
realizar del hardware y por la posibilidad de que existan múltiples máquinas
virtuales en ejecución en un mismo sistema real.

Las principales ventajas de esta solución es que permite realizar una protección
por aislamiento de los recursos del sistema, ya que el sistema virtualizado solo
verá dispositivos virtuales y en caso de cualquier problema solo podrá afectar a
la máquina virtual quedando la máquina real protegida. Adicionalmente son un
medio ideal para la realización de pruebas de sistemas operativos, como la
prueba de módulos en desarrollo o la prueba de servicios que se desean
implementar en una máquina real. También permiten aprovechar mejor el hardware
disponible, entregando servicio en un mismo equipo a diferentes sistemas
operativos que en conjunto comparten de forma eficiente el hardware disponible.
